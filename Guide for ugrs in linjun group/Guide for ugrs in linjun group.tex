%!TEX program = xelatex
% 使用 ctexart 文类,UTF-8 编码
\documentclass{article}
  \usepackage{xeCJK,indentfirst}
  \usepackage{amsfonts, amsmath, amssymb}
  \usepackage{graphicx}
  \usepackage[normalem]{ulem}

    \newtheorem{theorem}{Theorem}[section]
  \newtheorem{lemma}{Lemma}  
  \newtheorem{definition}{Definition}[section]
  \newtheorem{proof}{Proof}[section]  
  \numberwithin{equation}{section}

    \newcommand{\bra}[1]{\langle #1 |}
  \newcommand{\ket}[1]{| #1 \rangle}
  \newcommand{\bracket}[2]{\langle #1 | #2 \rangle}
  \newcommand{\bracketl}[3]{\langle #1 | #2 | #3 \rangle}
  \newcommand{\func}{\mathrm \,}
  \newcommand{\define}[2]{
   \begin{definition}
    \begin{description}
    \item[#1]
    #2
   \end{description}
   \end{definition}
  }
  \newcommand{\mean}[1]{\langle #1 \rangle}

  \newcommand{\sch}{Schr\"odinger} 
  \newcommand{\grad}{\nabla}

  \setlength{\parindent}{2em}
  \setlength{\textheight}{240mm}
  \setlength{\textwidth}{155mm}
  \setlength{\oddsidemargin}{0mm}
  \setlength{\evensidemargin}{0mm}
  \setlength{\topmargin}{-20mm}
  \renewcommand{\baselinestretch}{1.2}
  \title{王林军老师课题组本科生入门指南}
  \author{Chaoqun ZHANG-张超群\\Rui LI - 李睿}
  \date{\today}
  \begin{document}
    \maketitle
    \tableofcontents
    \newpage
    \section{引言}
  \begin{quote}
    老师说``这个东西很简单''的时候,往往是普通化学本科生搞不定的东西。
    \begin{flushright}
      ————李睿,在抱怨化学本科生编程能力不够时的调侃
    \end{flushright}
  \end{quote}
  \begin{quote}
    你有问题就要问,你如果不问,我就默认你都会了。
    \begin{flushright}
      ————王林军老师
    \end{flushright}
  \end{quote}

  欢迎进入王林军老师的课题组!无论您是由于何种原因,出于什么目的,来到王林军老师的课题组足以证明了您的勇气!\sout{看,在王林军老师组的都很休闲的}

  由于王林军老师研究比较``底层''的计算化学,所以会对数学、物理、计算机等知识内容会有更高的要求。而历届在王老师做事的本科生们都不止一次地抱怨来王老师的课题组的门槛非常高,而且没有合适的入门指南,使得难度进一步提高(\sout{你会发现你听了一年组会都学不到什么东西}),同时化学系对数理方面的要求实在不敢恭维,从而刚参加王老师的组会的时候会近乎100\%地出现``这是什么/我是谁/我在干什么''的疑惑,因此编者们共同商量,觉得出一份入门指南非常的重要,于是草草编写了这样的一份。

  另外,本组有与之配套的入门训练,但是暂时由于种种原因没有对每个本科生展开,所以在阅读本指南时,也可以花时间接触本组关于代码书写(行星运动)和势间跳跃方法(Tully文章)的
  基本训练,关于后者我们在指南中略有涉及,但是具体的代码实现还要靠自己的练习才行。由于并非专业科研文章,此指南在参考文献方面会比较欠缺,但是偶尔举出的几篇文章仍然
  推荐读者阅读相关内容,对理解本指南和王老师课题组的工作都有一定帮助。本指南将主要为王老师组的非绝热动力学方向服务,王老师课题组还有其他的研究方向,
  比如全局优化问题,机器学习相关和固体材料性质的计算等,这些内容在本指南中不作专门的讨论。
  
  简而言之,本指南旨在为诸位不同时间加入王林军老师课题组或者感兴趣的同学们提供一个深入了解的机会,因此很多内容不追求深度,编者认为,对于决定在这里完成毕设或者
  深造的同学来说,本指南只是一个入门皮毛。另外,编者们水平有限,如果有错误也是正常现象,望读者海涵。

    \section{Linux基础及服务器使用}
    Linux和macOS、Windows并称为三大电脑操作系统,都是用来完成用户和计算机之间的互动。Linux由于其出色的稳定性以及免费(相对于Unix系统),被广泛用于服务器的操作系统。当然Linux是存在图形界面的,但它的最为使用,也最为核心的部分还是它的终端,也就是字符界面。为了如何操作这玩意儿,一些基础知识是必须的。

    \subsection{连接至服务器}
    王林军老师课题组是使用浙大西溪校区的超级计算机集群来进行日常的工作的,简单地就叫服务器。听起来非常高大上,我们所编写的程序(或者商业软件)在普通计算机上当然也可以运行,只是很多时候
    需要过分长的时间,还要保持电脑全天开机,基本是不可能的,因此我们需要交给集群处理。

    集群计算机的操作系统都是Linux系统,我们需要用自己的个人电脑连接到服务器上,这样可以实现对服务器的远程操作,如果你的个人电脑是Linux或者mac系统,可以直接
    在终端使用ssh命令登录服务器,但是由于组内工作电脑和多数个人电脑为Windows系统,这一部分我们暂时不作展开,有兴趣的同学可以直接网上搜索。在Windows系统下连接
    Linux服务器,需要通过一些软件的辅助,比如PuTTY和XSHELL,鉴于后者有不少优势,比如更加新人友好\sout{并且好看},我们就以后者举例。

    XSHELL目前有官方的中文网站\footnote{https://www.netsarang.com/zh/xshell/},可以找到学生用的免费版,下载XSHELL和XFTP两个软件,
    前者是一个在Windows操作系统下实现ssh功能的软件,用于连接远程服务器;后者是实现sftp功能的软件,用于服务器和本地计算机之间的文件传递。
    这两个软件可以满足在理论计算化学组连接远程服务器的一切需求。安装好之后打开XSHELL,如下图所示新建会话,
    \begin{center}
      \includegraphics[width = 10cm]{fig/xshell1.jpg}
      \includegraphics[width = 8cm]{fig/xshell2.jpg}
    \end{center}
    随便起个名字,主机填入10.22.51.200(注意:想要成功连接该IP需要浙大内网:ZJUWLAN、有线网络或者RVPN均可),端口号默认22,在用户身份验证界面
    用户名输入ugrs3\_LJ,密码为linjungroup\footnote{这个是专为入门本科生注册的账号,主要供学习使用,毕设或者读研读博正式在组内工作会有新的账号}。
    此时如果见到类似下图的界面,就代表已经成功登陆进我们的服务器,并处于\textbf{主目录}下。
    \begin{center}
      \includegraphics[width = 10cm]{fig/xshell3.jpg}
    \end{center}
    登录服务器之后,就可以按照下一小节的内容进行操作,学习Linux系统的各类指令,和如何在Linux系统下编辑文件,
    为了安全起见,我们建议每个读者在主目录建立自己的文件夹,然后再自己的文件夹中进行各种操作,避免对他人的文件误操作而造成损失。

    \subsection{Linux基本操作}
    \begin{description}
    \item[cd]
      进入输入目录的文件夹;
      `cd folder/'
      P.S : `cd ..' 返回上一级文件夹
    \item[ls]
      列出所在文件夹的子文件夹;

    \item[vi / vim]
      打开文件查看内容;vim事实上是一个文本编辑器(比方说这个文档的一部分也是用vim编辑的)

      P.S :输入不存在的文件可直接创立该文件;

    vim有一个梗,就是几乎所有的人刚上手的时候都不知道{\textbf 该怎么退出}。在这里面稍微描述一下最基本的使用方法:
    \begin{description}
      \item[i / Ins] 如果已经打开了某个文本,那么这两个键中的一个能够允许你对文本作出修改。它左下角会显示 ``--INSERT--'' 这样的文段,这说明你处在编辑模式,这时候你按出的大部分操作会如同你使用大部分文本编辑器的时候一样,会完整地呈现在你正在修改的文本上。
      \item[Esc] 无论你处在何种模式,按了Esc你就能回到正常模式,在该模式下你可以用它自身的快捷键来直接作出一些修改,也可输入命令来完成别的操作(比如{\textbf 退出})。

      \item[:q!] 这里`:'表示进行对vim程序自身的命令,`q'是quit,`!'表示强制。也就是输了这个命令并按回车后会强制退出并不保存文件。

      \item[:wq] 以此类推,这个表示的是保存并退出。没错,`:w'就是进行保存,并不退出。
    \end{description}
    好了,现在你也是会用vim的人了!跟我一起喊,\sout{vim天下第一!}
    \item[pwd]
      显示所在目录路径;它会显示所处目录的绝对路径,而这个绝对路径是你在任何别的目录下都能访问的。从该角度出发,事实上别的几乎所有操作都可以指定绝对路径(比如简单的访问,或者复制,或者在程序里面调用/生成文件)。

    \item[mkdir]
      在所在目录下建立文件夹:
      `mkdir levest' 表示建立名字为`levest'的文件夹

    \item[cp] 复制文件至某个目录
      `cp einfield.com slot' 表示将`einfield.com' 在`slot'文件夹下进行粘贴

    \item[mv] 原则上它表示将某个东西移动到什么地方,但它的另一个神奇操作在于它可以重命名文件/文件夹(后缀还是要加好)。

    \item[rm] \textbf{ 危险!}这是表示删除操作。

    \item[-r] 在Linux中大部分的`-*' 表示了所进行的主要命令的附加指令(附加属性,\sout{buff})。像这里`-r'一般用来表示是对整个文件夹进行操作,像上面的cp, rm 都可以加上-r。

    \item[*] *一般在Linux表示缺省符号,也就是在这里面可以填任何东西。比方说,`rm *.c' 表示任何后缀为`.c'的文件都会被删除\footnote{这就是为什么sudo rm -rf /* 这个梗会流行的原因。}。

    \end{description}

    读者可以在登录之后所在的主目录下先输入ls和回车,看到当前目录下的文件和文件夹,然后输入“mkdir 文件夹名”的方式创建属于自己的文件夹,然后cd自己的文件夹,之后再这个目录
    下进行后续的操作。
    \subsection{使用服务器递交任务}
    我们的超算集群,简单地说是由很多的CPU(\textbf{节点})构成的,每个CPU有28个\textbf{核},据说服务器的单核运行效率其实是低于正常个人计算机的,但是优势在计算资源非常多。
    西溪校区的超算集群主要是王林军老师和洪鑫老师出资,也主要是这两个课题组在使用。王老师占有其中的48个节点,共1344个核。
    这些节点按照一定的顺序组合成了两个\textbf{队列},分别为ljcluster和ljtest,前者又可以分为wenchang和quantum,使用ljclustat指令可以看到
    全组的节点使用情况。尽管是为本科生学习练习使用的账号,ugrs3\_LJ账号仍然有对服务器三个节点的访问权限(即ljtest队列)。
    我们在服务器上递交任务,就是将自己的程序交给这些核去运算,要实现这个靠的是PBS任务管理系统。
    
    TODO

    \section{数学基础}
    \subsection{线性代数回顾及变分法}
    数学基础的第一小节将关注于最基础的数学内容,主要是线性代数的一些知识,由于长时间不用可能有所生疏,
    我们就在这里作简单的回顾,并且引入变分和线性变分法的概念,熟悉这部分内容的读者可以直接跳过。

    TODO
    \subsection{符号使用}
    在量子力学相关领域,人们使用``左矢''(bra)和``右矢''(ket)来描述一个``状态''。我们可以先简单地认为右矢代表了一个列向量,左矢则代表了一个行向量。它们是这么表示的:
    \[
    | \xi \rangle \rightarrow \text{右矢}, \langle \xi | \rightarrow \text{左矢}    
    .\]  
    我们知道,一个行向量和列向量相乘能够得到一个实数,这也可以理解为两个行向量进行内积的过程,比如两个右矢$| \xi_1 \rangle $ 和$| \xi_2 \rangle $ ,此时它表示为
    \[
    \langle \xi_1 | \xi_2 \rangle 
    .\] 
    当然量子力学远不止于此,它有可能包括了各种奇奇怪怪的东西,使得它并不能完全用向量来描述,比如在线性代数里我们有时候需要用矩阵,在量子力学里它们统统被概括为算符。算符是作用在矢量上的。如果你学过张量分析/矢量代数,你还可以认为它代表了一个张量。有些人们喜欢用$\hat{Q}$类似这样的形式来表达一个算符(也就是加个帽子),也有些人喜欢用粗体,比如$\mathbf{Q}$这样来表达,也有人不加任何东西\footnote{没错就是发明这一套符号系统的\Large{Dirac}大佬。},只要它``在该在的地方''就可以识别为一个算符。算符放在左矢的右边,或者右矢的左边,或者左矢和右矢的中间,也就是
    \[
    \hat{Q}| \xi \rangle ,\langle \xi | \hat{Q}, \langle \xi_1 | \hat{Q} | \xi_2 \rangle   
    .\] 
    到了这里,当然就会出现作用顺序(方向)的问题。不过其实这一套和线性代数非常相像\footnote{事实上这一套就是线性代数——也就是海森堡开发的矩阵力学的核心,量子力学就是线性代数(并不是)},算符默认是作用在右边的,如果需要作用在左矢,那么对应的算符为原算符的厄米共轭(Hermite Conjugate),用$\hat{Q}^\dagger$来表达。那么某种程度上,左矢代表的是右矢的共轭转置,左矢和右矢相作用得到的是两者的内积。

    当然这么说可能很难有感觉这些到底是什么东西,让我们再换一种方式来表达。算符,变换,这些东西实则上都代表了类似于函数的概念,也就是传进去一个东西,传出去另一个东西,这两个东西可以相等,可以完全不是同一类。而左矢和右矢则代表了可以传进去的``参量'',并经过算符作用后得到了另一个左矢或右矢。比如一个关于$x$的函数$f(x)$,经过求导算符$\frac{d }{d x} $作用后得到它的导函数$f'(x)$,实则上也没有脱离这样的描述体系\footnote{从而为Dirac说明海森堡的矩阵力学和薛定谔的波动力学是等价表述做好了铺垫。}。

    \subsection{完备基展开}
    完备基展开是一个非常重要的概念,它直接提供了解决众多微分方程的方法,而解决微分方程是量子力学的至关重要的一个内容。

    ``完备''一词说明某一个基组的完备性,也就是我只要通过这个基组的线性组合就能完整描述我所研究的空间中的所有向量。我用$\hat{x}$,$\hat{y}$,$\hat{z}$三个基向量就能完整描述三维空间里的所有向量,我用${x^n}$便能描述一维的所有函数,我用${\sin{nx},\cos{nx}}$就能描述有限空间内的函数/周期性函数,这些都构成了所谓的完备基。

    在这里请允许编者引入施图姆-刘维尔型方程和对应本征值问题。
    \begin{theorem}
    	对形如
	\begin{equation}
		\frac{d }{d x} \left[ k(x) \frac{d y}{d x}  \right] -q(x)y + \lambda \rho(x) y = 0 \quad (a\leqslant x \leqslant b)
	\end{equation}
	的方程,若$k(x),k'(x),q(x)$ 连续或最多以$x=a$ 和$x=b$ 为一阶极点\footnote{也就是最低项为$\frac{1}{x}$ 项。},则
	\begin{enumerate}
		\item 存在无穷多个本征值
		\[
		\lambda_1 \leqslant \lambda_2 \leqslant \lambda_3 \leqslant ...
		,\] 
		相应地有无限多个本征函数
		\[
			y_1(x),y_2(x),y_3(x),...
		.\] 
		这些本征函数的排列持续正好使节点个数依次增多(即函数值为零的点的个数)。
		\item 所有本征值 $\lambda_n \geqslant 0$,
		\item 相应于不同本征值$\lambda_m$ 和 $\lambda_n$的本征函数$y_m(x)$和$y_n(x)$在区间$[a,b]$上带权重$\rho(x)$正交,即
			\begin{equation}
				\int ^b_a y_m(x) y_n(x) \rho(x) \, dx = 0 
			\end{equation}
		\item 本征函数族$y_1(x),y_2(x),y_3(x)...$是完备的,即函数$f(x)$如具有连续一阶导数和分段连续二阶导数,且满足本征函数族所满足的边界条件,就可以展开为绝对且一致收敛的级数
			\[
				f(x) = \sum_n f_n y_n(x)
			.\] 
	\end{enumerate}
    \end{theorem}
    并由此引入广义傅里叶级数展开
    \[
	    \int ^b_a f(x) y_m(x) \, dx = \sum_n \int f_n y_m(x)y_n(x) \, dx = f_m \int y_m(x)^2 dx   
    .\] 
    即
    \[
	    f_m = \frac{\int ^b_a f(x) y_m(x) \, dx }{\int y_m(x)^2 \, dx }
    .\] 
    这一套系统非常的重要,它能够直接用来解决球坐标下的稳态问题(使用Legendre函数),柱坐标下的稳态问题和动态演化问题(使用贝塞尔函数),以及球坐标下的时间演化问题(使用球贝塞尔函数),其核心内容就在于使用广义傅立叶级数进行展开得到精确解并同时满足边界条件。

    没什么感觉?在这里举一个非常典型的例子,二维传热稳态问题,
    \begin{quote}
	    均匀的薄板占据区域$0<x<a,0<y<\infty$,边界上的温度
	    \[
		    u\big|_{x=0} = 0, u\big|_{x=a} = 0, u\big|_{y=0} = u_0, \lim_{y \to \infty} u =0
	    .\] 
	    求解板的稳定温度分布。
    \end{quote}
    传热这东西,它满足一个扩散方程
    \begin{equation}
	    \frac{\partial u}{\partial t} -a^2 (\frac{\partial ^2 u}{\partial x^2 } + \frac{\partial^2 u}{\partial y^2} + \frac{\partial ^2 u}{\partial z^2}  ) = 0
    \end{equation}
    稳态,也就是它不随时间变化,$\frac{\partial u}{\partial t} = 0$,而这是二维,所以
    \begin{equation}
	    \frac{\partial ^2 u}{\partial x^2 } + \frac{\partial ^2 u}{\partial y^2 } = 0
	    \label{2Ddiffusion}
    \end{equation}
    先分离变量,设$u=X(x)Y(y)$,代入(\ref{2Ddiffusion}),左右同除以$u$,
    \begin{equation}
	     \frac{d ^2 X}{X d x^2} = - \frac{d ^2 Y}{Y d y^2} = -k^2 
    \end{equation}
    由于左边只关于$x$,右边只关于$y$,那么它们只能等于一个常数,考虑到在$x$方向上的边界条件(两边为零,不应该是指数型函数),它们应等于一个负数$-k^2$.
    直接解这个常微分方程,得到
    \begin{equation}
    	\begin{cases}
		X = A_k\sin{kx} + B_k\cos{kx} \\
		Y = C_k e^{-kt}
    	\end{cases}
    \end{equation}
    由于这个世界实在不太可能有无穷大的温度,故$Y = e^{kt}$项不考虑。

    考虑到$u\big|_{x=0} = u\big|_{x=a} = 0$,$k$应该要满足某种条件才能满足这个,那么事实上
    \begin{equation}
	    X = A_n \sin{\frac{n\pi x}{a}} , k = \frac{n\pi x}{a}, n \in N_+
    \end{equation}
    由于前面提到的本征值问题,$-k^2$是$X$的常微分方程的本征值,对应本征函数构成完备基,对$Y$也有相应情况,从而$u$可以展开为$XY$的线性组合,即
    \begin{equation}
	    X = \sum_n A_n \sin{\frac{n\pi x}{a}} e^{- \frac{n\pi y}{a}}
    \end{equation}
    可是……$A_n$怎么确定呀?
    
    就是广义傅立叶级数呀!
    注意到$u\big|_{y=0} = u_0$,有
    \begin{equation}
	    A_n = \int ^a_0 u_0 \sin{\frac{n\pi x}{a}} \, dx = \frac{a u_0}{n\pi} (1-\cos{n\pi}) 
    \end{equation}
    于是我们成功地获得了温度分布的解析解!(?)\footnote{当然有些人会觉得级数展开不算解析解。能展开成简单函数求和已经不容易了\sout{饶了我吧}}

    花了这么多篇幅,只是想要说明完备基展开是一个非常重要的内容,它提供了一个解决复杂的微分方程问题的方法\footnote{其实线性代数里的本征值和本征向量也有同样的性质。}。同时它从某种角度上已经道明了量子力学的一个重要的点:
    \begin{center}
    	所有的态都能用本征态的线性组合来表示。
    \end{center}

    哦,我们的读者,然而故事还没有结束呢。

    我们先用曾经描述的量子力学符号来说说这个完备基。对于某个完备的特征值集$\{\lambda_n\}$和特征向量集$\{| e_i \rangle \}$,我们认为特征向量集已经正交归一\footnote{即不同的两个特征向量之间相互正交,且每个向量模为1(即\sout{自交}自己与自己的内积为1)。你可能还记得存在多重特征根的情况,此时你或许也记得这个特征根对应多个特征向量,它们并不一定需要正交。那么你想想就能明白,你能通过某些奇技淫巧让它们互相之间变得正交并仍旧是特征向量(比如大名鼎鼎的Gram-Schmidt 正交化)。},那么有定理
    \begin{theorem}
	\begin{equation}
		\sum \ket{e_i}\bra{e_i} = 1 \, / \, \int dq' \, \ket{q'}\bra{q'} = 1
	\end{equation}
    \end{theorem}

    \sout{编者自我检讨:写的太快了}

    $| e_i \rangle \langle e_i | $是一个全新的操作,事实上它其实就是一个算符。为什么呢?因为你可以看到,
    \begin{equation}
	    (| e_i \rangle \langle e_i | ) | \xi \rangle = | e_i \rangle \langle e_i | \xi \rangle = \langle e_i | \xi \rangle | e_i \rangle  
    \end{equation}
    这是由于$ \langle e_i | \xi \rangle $是一个数,所以它可以游荡在任何地方(向量可以非常自然地乘上任何一个数而不需要管顺序)。也就是我作用于$| \xi \rangle $然后获得了一个新的向量,那么它就是一个算符。

    $1$是沿用了Dirac大佬的写法\footnote{编者表示被Dirac虐得越多\sout{对Dirac越爱得深沉}},就是表示单位算符(或者单位矩阵),作用后得到原来的结果,即
    \begin{equation}
    	1 | \xi \rangle = | \xi \rangle 
    \end{equation}
    所以上面的定理就是在说我把特征向量按照那种方式组合求和后它就会变成单位算符。

    \begin{proof}
    	证明思路:验证其对完备基中任意向量都成立$\Rightarrow$空间内任意向量皆成立

\begin{equation}
	\sum_i \ket{e_i}\bracket{e_i}{e_j} = \sum_i\ket{e_i} \delta_{ij} = \ket{e_j}
\end{equation}
从而说明该算符对所有特征向量都满足自身为单位算符。

利用展开的唯一性(即一个向量只有一种方式展开/系数组是唯一的),得到
\begin{equation}
\ket{\psi} = \sum c_i \ket{e_i} \Rightarrow c_i = \bracket{e_i}{\psi}
\end{equation}
因此对所有向量皆满足自身为单位算符。
    \end{proof}
    恭喜你发现了\sout{镇站之宝},展开系数为
    \begin{equation}
    	c_i = \langle e_i | \psi \rangle 
    \end{equation}
    是不是有那么一点感觉?其实就是广义傅立叶级数展开。

    oh忘记说了,$\delta_{ij}$这是(离散)狄拉克函数,它表示$i=j$时其值为1,$i\neq j$时为零。同样地,对于连续狄拉克函数$\delta(x)$,表示$x=0$时其值不为零(或者正无穷),$x\neq 0$时恒为零,同时保证
    \begin{equation}
	    \int \delta(x) \, dx = 1 
    \end{equation}
    由此可以说明,
    \begin{align}
\delta(-x) & = \delta(x) \\
x\delta(x) & = 0 \\
\delta(a x) & = a^{-1} \delta(x)\\
\delta(x^2-a^2) & = \frac{1}{2} a^{-1} {\delta(x-a)+\delta(x+a)} \quad (a>0)\\
\int \delta(a-x) \, dx \, \delta(x-b) & = \delta(a-b) \\
f(x) \delta(x-a) &=  f(a)\delta(x-a)\\
\int f(x)x\delta(x) \, dx & = 0
\end{align}

    \subsection{傅立叶变换}
  \begin{quote}
    它,改变了世界。
    \begin{flushright}
      ——李睿,谈傅立叶变换时
    \end{flushright}
  \end{quote}     
  咕咕……咕咕咕……

  \section{量子力学}
  本指南默认入门者对量子力学最基础的部分有一定的理解,因此不会从Schr\"odinger方程入手,也不会涉及五大基本假设。另外,本指南也不涉及高深内容,只有在电子结构部分会从实用的角度简单介绍电子(费米子)的二次量子化。如果你需要量子力学的基础知识,暂时可以参阅各类量子力学教材,但是各类书籍难度差别巨大,我们为初学者推荐Griffiths的量子力学导论,基本看过基础理论部分就已经能满足我们的多数需求。另外,可以参见Levine量子化学中的量子力学基础部分,推导详实。

  \subsection{概念总结篇}
    \begin{enumerate}
    \item 算符对易子:
  \begin{equation}
  [A,B] = AB - BA
  \end{equation}

  \item 算符对易常用公式:
  \begin{align}
  [A,BC]& = [A,B]C - B[A,C]\\
  [AB,C]& = [A,C]B + A[B,C]
  \end{align}

  \item $x$与$p$的对易关系:
  \begin{equation}
  [x,p]\psi=(xp-px)\psi = -i\hbar[x \frac{\partial}{\partial x}\psi-\frac{\partial}{\partial x}(x\psi)]= i\hbar \psi
  \end{equation}

  \item 波函数是量子态在基组中的投影:
  \begin{equation}
  \psi (x) = \langle x | \psi \rangle , \quad \psi(p) = \langle p | \psi \rangle
  \end{equation} 
\end{enumerate}
  

\begin{theorem}
Ehrenfest Theorem:

\begin{align}
\mean{\textbf{p}}= m \frac{d \mean{\textbf{x}}}{dt}\\
\mean{\textbf{F}}= \frac{d\mean{\textbf{p}}}{dt}
\end{align}
\end{theorem}

\begin{proof}
  充分考虑到
  \begin{equation}
  \frac{d\ket{\phi}}{dt}=\frac{1}{i\hbar}[\frac{\textbf{p}^2}{2m}+V(\textbf{x})]\ket{\phi}
  \end{equation}

  \begin{align*}
  \frac{d \mean{\textbf{x}}}{dt}  & = \frac{d \bracketl{\psi}{\textbf{x}}{\psi}}{dt} \\ 
  & = \frac{d\bra{\psi}}{dt} \textbf{x} \ket{\psi} + \bra{\psi} \textbf{x} \frac{d\ket{\psi}}{dt} \\
  &= - \frac{1}{i \hbar}\bra{\psi}[\frac{\textbf{p}^2}{2m}+V(\textbf{x})]\textbf{x}\ket{\psi}+ \bra{\psi}\textbf{x}\frac{1}{i\hbar}[\frac{\textbf{p}^2}{2m}+V(\textbf{x})]\ket{\psi}\\
  &= -\frac{1}{i\hbar}(\frac{\textbf{p}^2}{2m}\textbf{x}-\textbf{x}\frac{\textbf{p}^2}{2m})\ket{\psi}\\
  &= -\frac{1}{2mi\hbar}\bracketl{\psi}{[\textbf{p}^2,\textbf{x}]}{\psi}\\
  &=-\frac{1}{2mi\hbar}\bracketl{\psi}{2\textbf{p}[\textbf{p},\textbf{x}]}{\psi}\\
  &=\frac{1}{m}\bracketl{\phi}{\textbf{p}}{\phi}\\
  &=\mean{\textbf{p}}
  \end{align*}

  \begin{align*}
  \frac{d\mean{\textbf{p}}}{dt} &= \frac{d\bracketl{\psi}{\textbf{p}}{\psi}}{dt}\\
  &=\frac{d\bra{\psi}}{dt} \textbf{p} \ket{\psi} + \bra{\psi} \textbf{p} \frac{d\ket{\psi}}{dt} \\
  &= - \frac{1}{i \hbar}\bra{\psi}[\frac{\textbf{p}^2}{2m}+V(\textbf{x})]\textbf{p}\ket{\psi}+ \bra{\psi}\textbf{p}\frac{1}{i\hbar}[\frac{\textbf{p}^2}{2m}+V(\textbf{x})]\ket{\psi}\\
  &= -\frac{1}{i\hbar}\bracketl{\psi}{[V(\textbf{x}),\textbf{p}]}{\phi} \\
  & = -\frac{1}{i\hbar}\bracketl{\phi}{i\hbar\frac{\partial V(\textbf{x})}{\partial \textbf{x}}}{\phi}\\
  & = \bracketl{\phi}{[-\frac{\partial V(\textbf{x})}{\partial \textbf{x}}]}{\phi}\\
  & = \mean{\textbf{F}}
  \end{align*}

  \begin{align*}
  [V(\textbf{x}),\textbf{p}] \ket{\psi} & = [V(x),-i\hbar \frac{\partial}{\partial \textbf{x}}]\ket{\psi}\\
  & = -i\hbar V(x) \frac{\partial }{\partial \textbf{x}} + i\hbar \frac{\partial}{\partial \textbf{x}}[V(\textbf{x})\ket{\psi}]\\
  &= -i\hbar V(x) \frac{\partial }{\partial \textbf{x}}+ i\hbar \frac{\partial V(\textbf{x})}{\partial \textbf{x}}\ket{\psi} + i\hbar V(\textbf{x}) \frac{\partial \ket{\psi}}{\partial \textbf{x}}\\
  &= i\hbar \frac{\partial V(\textbf{x})}{\partial \textbf{x}} \ket{\psi}
  \end{align*}
  \end{proof}

  \begin{theorem}
    Hellmann-Feynman Theorem:
    对于依赖某个参数$\lambda$(在本指南的后文中通常为原子核坐标$\mathbf{R}$)的哈密顿算符$\hat{H}_\lambda$本征态$|\psi_\lambda\rangle$,其能量期望值对参数的求导,等于哈密顿量对参数求导再求期望,即
    \begin{equation}
      \frac{\mathrm{d}E_\lambda}{\mathrm{d}\lambda} = \langle\psi_\lambda|\frac{\mathrm{d}\hat{H}_\lambda}{\mathrm{d}\lambda}|\psi_\lambda\rangle
    \end{equation}
  \end{theorem}

  \begin{proof}
    由于$|\psi_\lambda\rangle$是哈密顿算符本征态
    \begin{equation*}
      \hat{H}_\lambda|\psi_\lambda\rangle=E_\lambda|\psi_\lambda\rangle
    \end{equation*}  
    容易得到
    \begin{align*}
      \begin{aligned}
        \frac{\mathrm{d} E_{\lambda}}{\mathrm{d} \lambda}=&\frac{\mathrm{d}}{\mathrm{d} \lambda}\left\langle\psi_{\lambda}\left|\hat{H}_{\lambda}\right| \psi_{\lambda}\right\rangle\\
        =&\frac{\mathrm{d} \langle\psi_\lambda|}{\mathrm{d} \lambda}\hat{H}_\lambda|\psi_\lambda\rangle+\langle\psi_\lambda|\hat{H}_\lambda\frac{\mathrm{d} |\psi_\lambda\rangle}{\mathrm{d} \lambda}+\langle\psi_\lambda|\frac{\mathrm{d}\hat{H}_\lambda}{\mathrm{d}\lambda}|\psi_\lambda\rangle\\
        =&E_\lambda\frac{\mathrm{d} \langle\psi_\lambda|}{\mathrm{d} \lambda}|\psi_\lambda\rangle+E_\lambda\langle\psi_\lambda|\frac{\mathrm{d} |\psi_\lambda\rangle}{\mathrm{d} \lambda}+\langle\psi_\lambda|\frac{\mathrm{d}\hat{H}_\lambda}{\mathrm{d}\lambda}|\psi_\lambda\rangle\\
        =&E_\lambda\frac{\mathrm{d}}{\mathrm{d}\lambda}\left\langle\psi_{\lambda} | \psi_{\lambda}\right\rangle+\langle\psi_\lambda|\frac{\mathrm{d}\hat{H}_\lambda}{\mathrm{d}\lambda}|\psi_\lambda\rangle\\
        =&\langle\psi_\lambda|\frac{\mathrm{d}\hat{H}_\lambda}{\mathrm{d}\lambda}|\psi_\lambda\rangle
      \end{aligned}
    \end{align*}
  \end{proof}

  \begin{description}
	\item[变分法] 在有限空间中寻找最优解;任意波函数对应的能量期望值都不小于最低本征态能量

	\begin{theorem}
	The eigenstate of the system has the minimum energy compared to all other states.
	\end{theorem}

	\begin{proof}
	\begin{align}
	&\begin{cases}
	H\ket{\psi_i} = E_i\ket{\psi_i}\\
	\ket{\psi}=\sum_i c_i \ket{\psi_i}
	\end{cases}\\
	\Rightarrow & \bracketl{\psi}{H}{\psi} = \sum_i c_i^* \bracketl{\psi_i}{H \sum_j c_j}{\psi_j} \\
	=& \sum_{ij}c_i^*c_j E_j \delta_{ij}=\sum_i c_i^* c_i E_i \geq \sum_i c_i^* c_i E_{0}=E_{0}
	\end{align}
	\end{proof}

	\item[微扰法] 尽可能利用简单体系的精确解描述复杂问题
	\begin{equation}
	H\ket{\psi_i}=E_i\ket{\psi_i} \quad H=H^{(0)} +\lambda H^{(1)}
	\end{equation}

	\begin{align*}
	&\begin{cases}
	\ket{\psi_i} = \ket{\psi_i^{(0)}}+\lambda \ket{\psi_i^{(1)}}\\
	E_i=E_i^{(0)} + \lambda E_i^{(1)}
	\end{cases}\\
	\Rightarrow & (H^{(0)}+\lambda H^{(1)})(\ket{\psi^{(0)}_i}+\lambda \ket{\psi_i^{(0)}})=(E_i^{(0)}+\lambda E^{(1)}_i)(\ket{\psi_i^{(0)}}+\lambda \ket{\psi^{(1)}_i}) \\
	\Rightarrow & \begin{cases}
	(H^{(0)}-E_i^{(0)})\ket{\psi_i^{(0)}}=0\\
	(H^{(0)}-E_i^{(0)})\ket{\psi_i^{(1)}}+(H^{(1)}-E_i^{(1)})\ket{\psi_i^{(0)}} =0
	\end{cases}\\
	\Rightarrow & \bracketl{\psi_j^{(0)}}{H^{(0)}-E_i^{(0)}}{\psi_i^{(1)}}=\bracketl{\psi_j^{(0)}}{(E_i^{(1)}-H^{(1)})}{\psi_i^{(0)}}\\
	\Rightarrow & (E_j^{(0)}-E_i^{(0)})\bracket{\psi_j^{(0)}}{\psi_i^{(1)}}=E_i^{(1)}\delta_{ij} -\bracketl{\psi^{(0)}_j}{H^{(1)}}{\psi_i^{(0)}}\\
	\Rightarrow &
	\begin{cases}
	E_i^{(1)}=\bracketl{\psi_i^{(0)}}{H^{(1)}}{\psi_i^{(0)}} &\quad i=j \\
	\ket{\psi_i^{(1)}}=\sum_{j\neq i}\frac{\bracketl{\psi^{(0)}_j}{H^{(1)}}{\psi^{(0)}_i}}{E^{(0)}_i-E^{(0)}_j} &\quad i\neq j
	\end{cases}
	\end{align*}

	二阶微扰先略。

	微扰应用:

	\begin{equation}
	T=\frac{p^2}{2m}-\frac{p^4}{8c^2m^3}+....
	\end{equation}
	\begin{description}
		\item 非谐性:对势能项做微扰
		\begin{equation}
		V=V(0)+\frac{k}{2}x^2+\frac{\alpha}{6}x^3+\frac{\beta}{24}x^4+...
		\end{equation}
		\item Stark效应:对外场做微扰
		\begin{equation}
		H=H_0-\mu\cdot F
		\end{equation}
	\end{description}


\end{description}


\begin{description}
	\item[全同性原理] 量子世界的全同粒子不可区分,任何两个粒子交换不影响体系的状态;

	\item[交换算符]
	\begin{equation}
	\textbf{P}_{12}[f(q_1,q_2,....,q_n)]=f(p_2,p_1,....,p_n)
	\end{equation}
	\begin{equation}
	P_{12}[1s(1)\alpha(1)3s(2)\beta(2)]=1s(2)\alpha(2)3s(1)\beta(1)
	\end{equation}

	\item[交换算符的本征值]
	\begin{equation}
	\textbf{P}_{12}[\textbf{P}_{12}[f(q_1,q_2,.....,q_n)]]=f(q_1,q_2,...,q_n)\Rightarrow P_{12}^2=1
	\end{equation}
	Thus
	\begin{equation}
	c_i=\pm 1
	\end{equation}

	交换算符的本征值为实数,只能为$\pm 1$,对应波函数分别称为对称和反对称。不具备对称性的波函数无法用于描述全同粒子。

	\item[玻色子和费米子] 波函数反对称的粒子为费米子,而波函数对称的粒子为玻色子

	\item[反对称粒子的泡利不相容原理]
	\begin{equation}
	f(q_1,q_1,....,q_n)=-f(q_1,q_1,....,q_n) \Rightarrow f(q_1,q_1,....,q_n)=0
	\end{equation}
	\item[自旋统计] Pauli 使用量子场论证明了自旋统计理论,指出费米子(电子和质子等)拥有半整数自旋,而玻色子(光子和声子等)具有整数自旋,电子具有$\frac{1}{2}$的自旋。

	\item[相对论量子化学] 自旋角动量很自然地以内禀方式蕴含在相对论性狄拉克方程中。作为最低阶非相对论近似,薛定谔方程人为地丢弃了自旋这种相对论效应。

	\item[Slater行列式]
	\begin{equation}
	|\Psi(p_1,p_2,...,p_n)|=\frac{1}{\sqrt{N!}}
	\begin{vmatrix}
	\ket{\phi_1(q_1)} & \ket{\phi_2(q_1)} & ... & \ket{\phi_n(q_1)}\\
	\ket{\phi_2(q_2)} & \ket{\phi_2(q_2)} & ... & \ket{\phi_n(q_2)}\\
	... & ...& ... & ...\\
	\ket{\phi_1(q_n)} & \ket{\phi_2(q_n)} & ... & \ket{\phi_n(q_n)}\\
	\end{vmatrix}
	\end{equation}

	\item[双电子自旋]
	\begin{equation}
	\alpha(1)\alpha(2),\beta(1)\beta(2),\alpha(1)\beta(2),\beta(1)\alpha(2)
	\begin{cases}
	\frac{1}{\sqrt{2}}[\alpha(1)\beta(2)+\beta(1)\alpha(2)]\\
	\frac{1}{\sqrt{2}}[\alpha(1)\beta(2)-\beta(1)\alpha(2)]
	\end{cases}
	\end{equation}

\end{description}
  \begin{description}
	\item[库伦积分法]
	\begin{equation}
	\begin{cases}
	\ket{\phi_{1s}^{(1)}}=\frac{1}{\sqrt{\pi}}e^{-r_1}\\
	\ket{\phi_{1s}^{(2)}}\frac{1}{\sqrt{\pi}}e^{-r_2}
	\end{cases}
	\Rightarrow J=(1+\frac{1}{R})e^{-2R}
	\end{equation}

	\item[交换积分]
	\begin{equation}
	\begin{cases}
	\ket{\phi_{1s}^{(1)}}=\frac{1}{\sqrt{\pi}}e^{-r_1}\\
	\ket{\phi_{1s}^{(2)}}\frac{1}{\sqrt{\pi}}e^{-r_2}
	\end{cases}
	\Rightarrow K=(\frac{1}{R}-\frac{2R}{3})e^{-R}
	\end{equation}
	它在分子成键中起到了非常关键的作用,是量子效应。
\end{description}

\begin{description}
	\item[平均场近似] 忽略电子动态关联,多电子问题转化为单电子问题,多电子波函数简化为单电子波函数的乘积,使用自洽场得到单电子波函数

	\item[Koopmans 定理]假设离子轨道与中性分子轨道相同,可计算分子的电离能喝亲和势,改变电子数会引起单电子轨道重排和弛豫,与电子关联量级相似,但符号相反
\end{description}


\begin{description}
	\item[平均场近似] 忽略电子动态关联,多电子问题转化为单电子问题,多电子波函数简化为单电子波函数的乘积,使用自洽场得到单电子波函数

	\item[Koopmans 定理]假设离子轨道与中性分子轨道相同,可计算分子的电离能喝亲和势,改变电子数会引起单电子轨道重排和弛豫,与电子关联量级相似,但符号相反
\end{description}
\begin{align*}
&\mean{\hat{H}}_{Hartree} \\
=& \bracketl{\Psi(\{r_i\})}{\hat{H}}{\Psi(\{r_i\})}\\
=&(\prod_i\bra{\phi_i(r_i)})(\sum_k\hat{h}_k+\frac{1}{2}\sum_{k\neq l}\frac{1}{\hat{r}_kl})(\prod_j\ket{\phi_j(r_j)})\\
=&\sum_k\bracketl{\phi_k(r_k)}{\hat{h}_k}{\phi_k(r_k)}(\prod_{i\neq k}\bra{\phi_i(r_i)})(\prod_{j \neq k}\ket{\phi_j(r_j)})\\
&+\frac{1}{2}\sum_{k \neq l}\bracketl{\phi_k(r_k)\phi_l(r_l)}{\frac{1}{\hat{r}_{kl}}\phi_k(r_k)\phi_l(r_l)}(\prod_{i \neq \{k,l\}}\bra{\phi_i(r_i)})(\prod_{i \neq \{k,l\}}\ket{\phi_j(r_j)})\\
=&\sum_k\bracketl{\phi_k(r_k)}{\hat{h}_k}{\phi_k(r_k)}+\frac{1}{2}\sum_{k \neq l}\bracketl{\phi_k(r_k)\phi_l(r_l)}{\frac{1}{\hat{r}_{kl}}}{\phi_k(r_k)\phi_l(r_l)}
\end{align*}

\begin{description}
	\item[变分法与基组] Hartree-Fock 变分法中,波函数的形式原则上是任意的,但是波函数的随意变化数字处理起来非常麻烦,使用基组对波函数进行线性展开,对量子化学中的所有电子结构计算都至关重要。

	\item[原子轨道基组] 完备的基组在量子化学是很难实现的,从计算量的角度需要尽可能地减少基组的数目。氢原子的电子结构可解析求解,为理解更复杂原子和分子的电子结构打开了大门。使用氢原子和类氢离子的原子轨道作为基组,是解决复杂化学问题的通用做法,有着重要的意义。

	\item[原子轨道]
	\begin{equation}
		\psi(r,\theta,\phi)=R_{nl}(r)Y_{lm}(\theta,\phi)
	\end{equation}

	\item[Slater基组和Gaussian基组]
	\begin{equation}
	\chi(r)\sim e^{-\xi r} \qquad g(r)\sim e^{-\alpha r^2}
	\end{equation}

	从基组数目角度看,STO比GTO好;

	从波函数积分计算效率角度看,GTO比STO好。

	\item[高斯函数线性组合]
	\begin{equation}
	e^{-\xi r}=\frac{\xi}{2\sqrt{\pi}}\int_0^\infty \alpha^{-3/2} e^{-\xi^2/4\alpha}e^{-\alpha r^2} \, d\alpha
	\end{equation}
	\begin{equation}
	\Rightarrow \xi_\mu(r-R_A)\approx \sum_Pc_{p\mu}g_p(\alpha_{p\mu},r-R_p)
	\end{equation}
	\begin{equation}
	\begin{cases}
	g_{1s}(\alpha,r)=(8\alpha^3/\pi^3)^{1/4}e^{-\alpha r^2};\\
	(\alpha,r)=(128\alpha^5/\pi^3)^{1/4}xe^{-\alpha r^2};\\
	g_{3dxy}(\alpha,r)=(2048\alpha^7/\pi^3)^{1/4}xye^{-\alpha^2}
	\end{cases}
	\end{equation}
\end{description}

\begin{description}
	\item[Dunning 原子轨道基组] 如 aug-cc-pVDZ, aug-cc-pVTZ, aug-cc-pVQZ, aug-cc-pV5Z...

	aug:弥散, cc: correlation consistent; p: polarized; V: Valence bonds; DZ: double zeta.

	\item[赝势原子轨道基组] 当原子中的电子很多事,内层电子通常对原子性质的影响很小,但是其对应的全电子基组很复杂,为了减少计算量,其贡献可通过有效势能来描述,而且可以包含相对论效应的修正。

	\item[赝势基组]
	赝势基组包括赝势和基组两部分,内部电子的贡献采用赝势描述,直接放在哈密顿量里,外层价电子采用一般的基组。

	LANL1只考虑加电子,LANL2系列除了加电子外,还考虑次外层电子,因为他们与价层的能查不明显,而且对成键有贡献。

	LanL2DZ:对H-Ne使用D95V全电子基组,对Na-Bi使用赝势基组,也就是LANL有效核势加上DZ基组;LanL2DZ是常用基组,适合过渡金属等中等质量的金属元素。

	\item[总能量] 变分法保证基组越大,HF能量越小,越接近于精确值。

	\item[Hessian矩阵]
	\begin{equation}
	F_{ij}=\frac{1}{\sqrt{m_im_j}}\frac{\partial^2 U}{\partial x_i \partial x_j}
	\end{equation}

	\item[振动频率]
	\begin{equation}
	\sum_{j-1}^N (F_{ij}-\delta_{ij}\lambda_k)l_{jk}\Rightarrow det(F_{ij}-\delta_{ij}\lambda_{k})=0
	\end{equation}
\end{description}

\begin{description}
	\item[电子关联] Hartree-Fock理论中采用平均场处理电子-电子相互作用,缺乏电子关联,直接用多电子基组展开多电子波函数计算量巨大,很难实现。

	\item[如何有效加入电子关联] 以Hartree-Fock得到的单Slater行列式为基础,考虑多电子关联的主要方法有CI(Configurate Interaction)、MCSCF(Multi-Configured Self-Consistent Field)、MPn、CC(Couple Cluster)等。

	HF方法的计算量与电子数的四次方成正比,MP2为五次方,CISD和CCS为六次方,CCSD(T)为七次方,CISDT和CCSDT为八次方,随着电子数增加而迅速增加。

	\item[多电子组态] 在Hartree-Fock分子轨道的基础上,组态相当于多电子组态,用之可展开多电子波函数,对角化多电子哈密顿量,得到基态和激发态性质。

	若有K个分子轨道,每个轨道最多占据两个电子,总电子书数N,则n激发组态数为
	\begin{equation}
	C_N^nC_{2K-N}=\frac{N!}{n!(N-n)!}\frac{(2K-N)!}{n!(2K-N-n)!}
	\end{equation}

	每个组态都可以用Slater行列式表示。


	\item[多电子波函数的组态线性组合]
	\begin{align*}
	\ket{\Phi} &= c_0\ket{\psi_0} +\sum_{ra}c_a^r\ket{\psi_a^r}+\sum_{a<b,r<s}c_{ab}^{rs}\ket{\psi_{ab}^{rs}}+\sum_{a<b<c,r<s<t}c^{rst}_{abc}\ket{\psi^{rst}_{abc}}+...\\
	&=c_0\ket{\psi_0} + c_S\ket{S}+c_D\ket{D}+c_T\ket{T}+...
	\end{align*}

	Full CI(FCI)包括所有可能的激发,可以给出多电子问题的精确解,

	实际上即使我们只考虑有限个单电子基组,所有可能的N-电子基组数目也非常庞大。

	通常需要对组态进行阶段,只处理有限个N-电子基组。如只考虑单激发,成为CIS;截断到双激发,成为CISD。


	\item[组态相互作用关系] 单激发对基态能量没有直接贡献,双激发对基态能量的修正起首要作用,Hartree Fock的本征值问题等价于确保组态与单激发不直接混合,相差两个激发的组态之间没有直接相互作用。

	\item[Brillouin 定理]
	\begin{equation}
	\bracketl{\psi_0}{\hat{H}}{\psi_a^r}=\bracketl{a}{\hat{h}}{r}+\sum_b\bra{ab}\ket{rb}=\bracketl{\phi_a}{\hat{f}}{\phi_r}=0
	\end{equation}

	\begin{align*}
	\bracketl{\psi_0}{\hat{H}}{\psi_a^r}&=(\prod_i\bra{\phi_i})O_1(\prod_j\ket{\phi_j})\\
	&=\sum_k(\prod_i\bra{\psi{\phi_i}})h_k(\prod_l\ket{\phi_j})\\
	&=\sum_k \bracketl{\phi_k}{h_k}{\phi_k}*\prod_{i \neq k} \bra{\phi_i} \prod_{j \neq k} \bra{\phi_j}\\
	&-\sum_k \bracketl{\phi_k}{h_k}{\phi_k}
	\end{align*}
\end{description}

\begin{description}
	\item[Thomas-Fermi 理论]
		是现代密度泛函理论的雏形,能量期望值完全形成电子密度的泛函,使得电子结构处理变得非常简单。

		当时其精度有限,因为动能项通过简单的近似得到,而且没有考虑交换和关联对总能量的贡献。
	\item[Hohenberg-Kohn 定理]
	\begin{itemize}
		\item 所有客观测量的期望值原则上都是基态电子密度的泛函。

		电子密度分布$\rho(r)$决定了总电子数$N$,从而哈密顿量$H=T_e+U_{ee}+V_{ext}$总只有最后一项可能是不确定的。如果$\rho(r)$是两个不同哈密顿量$H$和$H'$的基态电子密度,而两者的基态波函数分别为$\ket{\Psi}$和$\ket{\Psi'}$,基态能量分别为$E_0$和$E'_0$,基态在不兼并的情况下有
		\begin{equation}
		\begin{cases}
		E_0&=\bracketl{\Psi}{H}{\Psi}\\
		&<\bracketl{\Psi'}{H}{\Psi'}\\
		&=\bracketl{\Psi'}{H'}{\Psi'}+\bracketl{\Psi'}{(H-H')}{\Psi'}\\
		&=E_0'+\int \rho(r)(V_{ext}-V'_{ext})\,dr\\
		E_0'&=\bracketl{\Psi'}{H'}{\Psi'}\\&
		<\bracketl{\Psi}{H'}{\Psi}\\
		&=\bracketl{\Psi}{H}{\Psi}+\bracketl{\Psi}{(H'-H)}{\Psi}\\
		&=E_0-\int \rho(r)(V_{ext}-V'_{ext})\,dr\\
		\end{cases}
		\end{equation}
		Then
		\begin{equation}
		E_0+E'_0<E_0+E_0'
		\end{equation}
		矛盾,故电子密度与唯一的哈密顿量对应。反过来,哈密顿量有唯一的基态电子密度。\footnote{在这里,需要指出的是此定理在基态不兼并的情况下成立。}因此,所有可观测量的期望值都是基态电子密度的泛函,特别地,基态总能量是电子密度的泛函:
		\begin{equation}
		E[\rho]=T_e[\rho]+V_{ext}[\rho]+U_{ee}[\rho]
		\end{equation}

		\item 基态电子密度原则上可以通过变分法严格得到。

	\end{itemize}

	\item[Kohn-Sham密度泛函理论]
	哈密顿量中的动能项难以用密度泛函直接表达,而波函数可以更容易地计算动能,为此1965年Kohn和Sham提出了融合波函数和密度的现代DFT理论,使密度泛函理论成为实际可行的理论方法。

	\begin{itemize}
		\item 动能
		\begin{equation}
		T_0[\rho]=\sum_i\bracketl{\phi_i}{-\frac{1}{2}\grad^2}{\phi_i}
		\end{equation}

		\item 库伦势
		\begin{equation}
		U_{cl}[\rho]=\frac{1}{2}\iint\frac{\rho(r')\rho(r)}{|r'-r|}\,drdr'
		\end{equation}
	\end{itemize}

\end{description}


\begin{theorem}
The external potential(and hence the total energy) $v(\textbf{r})$, is a unique functional of the electron density $n(\textbf{r})$.
\end{theorem}

\begin{proof}
The proof proceeds by \emph{reducio ad absurdum}. 

Assume that another potential $v'(\textbf{r})$, with ground state $\Psi'$ gives rise to the \emph{same} density $n(\textbf{r})$. Now clearly [unless $v'(\textbf{r})-v(\textbf{r})=\mathrm{const}$] $\Psi'$ cannot be equal to $\Psi$ since they satisfy different \sch equations. Hence, if we denote the Hamiltonian and ground-state energies associated with $\Psi$ and $\Psi'$ by $H$, $H'$ and $E$, $E'$, we have by the minimal property of the ground state,
\begin{equation}
E'=\bracketl{\Psi'}{H'}{\Psi'}<\bracketl{\Psi}{H'}{\Psi}=\bracketl{\Psi}{H+V'-V}{\Psi}
\end{equation}
so that
\begin{equation}
E'<E+\int [v'(\textbf{r})-v(\textbf{r})]n(\textbf{r})d\textbf{r}
\label{ueq1}
\end{equation}
Interchanging the primed and unprimed quantities, we find in exactly the same way that
\begin{equation}
E<E'+\int [v(\textbf{r})-v'(\textbf{r})]n(\textbf{r})d\textbf{r}
\label{ueq2}
\end{equation}
Addition of \ref{ueq1} and \ref{ueq2} leads to the incosistency
\begin{equation}
E+E'<E+E'
\end{equation}
Thus $v(\textbf{r})$ is (to within a constant) a unique functional of $n(\textbf{r})$; since, in turn, $v(\textbf{r})$ fixes $H$ we see that the full many-particle ground state is a unique functional of $n(\textbf{r})$.\cite{PhysRev.136.B864}
\end{proof}

\begin{theorem}
The functional that delivers the ground state energy of the system, gives the lowest energy if and only if the input density is the true ground state density.

For any positive integer $N$ and potential $v(\textbf{r})$, a density functional $F[n]$ exists such that
\begin{equation}
E_{(v,N)}[n]=F[n]+\iiint v(\textbf{r})n(\textbf{r})\,d^3\textbf{r}
\end{equation}
obtains its minimal value at the ground-state density of $N$ electrons in the potential $v(\textbf{r})$. The minimal value of $E_{(v,N)}[n]$ is then the ground state energy of this state.

\end{theorem}

\begin{proof}
Since $\Psi$ is a functional of $n(\textbf{r})$, so is evidently the kinetic and interaction energy. We therefore difine 
\begin{equation}
F[n(\textbf{r})] \equiv \bracketl{\Psi}{T+U}{\Psi}
\end{equation}
With its aid we difine, for a given potential $v(\textbf{r})$, the energy functional
\begin{equation}
E_v[n]\equiv \int v(\textbf{r})n(\textbf{r})d\textbf{r} + F[n]
\end{equation}
It can be easily proved that for a system of N particles, the energy functional of $\Psi'$
\begin{equation}
\varepsilon_v[\Psi']\equiv\bracketl{\Psi'}{V}{\Psi'}+\bracketl{\Psi'}{T+U}{\Psi'}
\end{equation}
has a minimum at the correct ground state $\Psi$, relative to arbitrary variations of $\Psi'$ in which the number of particles is kept constant. Let $\Psi'$ be the ground state associated with a different external potential $v'(\textbf{r})$. Then,
\begin{equation}
\varepsilon_v[\Psi']=\int v(\textbf{r})n'(\textbf{r})d\textbf{r}+F[n']>\varepsilon_v[\Psi]
\end{equation}
Thus the minimal property is established relative to all density functions $n'(\textbf{r})$ associated with some other external potential $v'(\textbf{r})$.\cite{PhysRev.136.B864}
\end{proof}


\begin{description}
	\item[电流密度]
	\begin{equation}
	j(r)=\frac{1}{2i}\sum_{i=1}^{N}[\grad_i\delta(r-r_i)+\delta(r-r_i)\grad_i]\quad f(r,t)=\bracketl{\psi(t)}{j(r)}{\psi(t)}\end{equation}


	 \item[海森堡方程]
	 \begin{equation}
	 \frac{\partial}{\partial t}j(r,t)=-i\bracketl{\psi(r,t)}{[j(r),H(r,t)]}{\psi(r,t)}
	 \end{equation}


\end{description}

  \section{量子化学-电子结构基础}

    \section{非绝热动力学与势间跳跃方法}
    简单的说,量子化学主要分为电子结构理论和动力学,上一大节专注于前者,本节就关注后者,电子结构计算可以给出电子能级的分布,激发态的位置以及其他一系列与之相关的静态性质,而动力学,正如其名,研究电子/原子核在特定情况下随时间的演化,通常化学中的光激发驰豫过程、电荷转移过程、异构化,甚至广义上的所有化学反应都是动力学的范畴。当然很容易想到,动力学的准确性是依赖于电子结构方法的准确性的,随着电子结构理论发展,动力学在研究化学物理等一系列问题中有不可忽视的作用。

    在Born-Oppenheimer近似下,我们可以针对特定的分子构型(核的位置)计算对应的电子能量,对于不同的核坐标$\mathbf{R}$(习惯上大写表示原子核坐标,小写表示电子坐标),可以给出一系列电子能量$U=U(\mathbf{R})$,这就是我们通常所说的势能面。需要注意的是,势能面是基于BO近似的结果,这是一切的前提。
    
    BO近似几乎是量子化学中最重要的近似条件,可以适用于我们平常遇到几乎全部静态问题和很多动态问题,BO近似下基态的势能面也是比较合理的。但是在特定的情况下,尤其是当研究的体系涉及多个激发态——比如光激发然后通过非辐射方式回到基态过程时,BO近似就显示出了其弊端。一类常见的BO近似失效的情况就是圆锥交叉(conical intersections),理想的圆锥交叉指的是两个势能面的至少两个维度在某一分子构型上简并(交叉)。通过早期实验光谱上和理论上的研究,人们发现涉及圆锥交叉的势能面的动力学是BO近似无法描述的。
    
    而此类问题又是化学和物理学中广泛存在的问题,为了解决这类动力学问题,必须把原子核和电子的运动一起加进动力学模拟的过程中,这类动力学通常称为非绝热动力学(non-adiabatic coupling)。处理核的运动有多种方式,经典的分子动力学就是只有原子核的运动,不属于非绝热动力学的范畴,但是经典运动是处理原子核的一种方式,另外还可以将原子核的运动也用量子力学求解,进行全量子的非绝热动力学模拟。王林军老师课题组常用的势间跳跃(surface hopping)方法,就是原子核做经典处理,电子做量子处理的一种混合量子经典的非绝热动力学方法。
      \subsection{非绝热耦合}
        作为“动力学”下的讨论,我们一定会用到含时Schr\"odinger方程:
        \begin{equation}
          i \hbar \frac{\mathrm{d} | \psi \rangle}{\mathrm{d} t}=\hat{H} | \psi \rangle
        \end{equation}
        或者是与之等价的量子Liouville方程:
        \begin{equation}
          \frac{\mathrm{d} \hat{\rho}}{\mathrm{d} t}=\frac{-i}{\hbar}[\hat{H}, \hat{\rho}]
        \end{equation}
        Liouville方程的优点是直接演化密度矩阵(TODO电子结构部分解释),动力学所关心的问题就是波函数/密度矩阵如何随时间演化,而在一般的量子力学语境中,波函数的演化通常是波函数系数的演化,含时的波函数可以写作不含时基组的线性组合
        \begin{equation}
          |\psi(\mathbf{r}, \mathbf{R}, t)\rangle=\sum_{j} c_{j}(t) |\phi_{j}(\mathbf{r} ; \mathbf{R})\rangle
          \label{linear combination}
        \end{equation}
        这个不含时基组可以是(也可以不是)定态Schr\"odinger方程的解,我们讨论一个更一般的情况,这个基组只被要求满足正交归一性,带入含时Schr\"odinger方程得到
        \begin{equation}
          \sum_j c_j \hat{H}|\phi_j\rangle=i \hbar \frac{\mathrm{d} | \psi \rangle}{\mathrm{d} t}=i \hbar\sum_j\left(\dot{c}_j|\phi_j\rangle+c_j\frac{\mathrm{d}}{\mathrm{d}\mathbf{R}}|\phi_j\rangle\dot{\mathbf{R}}\right)
        \end{equation}
        在上式左右两端乘以$\langle\phi_i|$就得到了
        \begin{equation}
          i \hbar \dot{c}_{i}=\sum_{j} c_{j}\left(V_{i j}-i \hbar \dot{R} \cdot d_{i j}\right)
          \label{time-depend-coeff}
        \end{equation}
        其中定义了$V_{ij}=\langle\phi_i|\hat{H}|\phi_j\rangle$和$d_{ij}=\langle\phi_i|\frac{\mathrm{d}}{\mathrm{d}\mathbf{R}}|\phi_j\rangle$,后者称为非绝热耦合(non-adiabatic coupling,NAC),该项所引发的问题是非绝热动力学的领域核心问题之一。在这里,我们可以通过另一种方式来理解它名字的来源以及它的物理意义:如果我们用最简单的数值方法计算非绝热耦合,即
        \begin{equation}
          d_{ij}=\langle\phi_i|\frac{\mathrm{d}}{\mathrm{d}\mathbf{R}}|\phi_j\rangle\approx\langle\phi_i(\mathbf{R})|\frac{|\phi_j(\mathbf{R}+\Delta\mathbf{R})\rangle-|\phi_j(\mathbf{R})\rangle}{\Delta\mathbf{R}}
          =\frac{\langle\phi_i(\mathbf{R})|\phi_j(\mathbf{R}+\Delta\mathbf{R})\rangle}{\Delta\mathbf{R}}
          \label{numerical dij}
        \end{equation}
        上式的最后结果说明非绝热耦合直接正比于不同核位置的波函数之间重叠,要知道如果这些态在同一个核坐标下满足正交性,当$i\neq j$时不应该有重叠。之所以有不等于0的非绝热耦合,就是因为在不同核坐标下的不同电子态间有“耦合”,而这个耦合正是超越BO近似所引入的,是“非绝热”而引入的。

        非绝热耦合还有另外的性质,它相当于算符$\frac{\mathrm{d}}{\mathrm{d}\mathbf{R}}$,要注意它并非量子力学中的力学量,这是因为它并非一个厄米算符,考虑到动量算符为$i\hbar\frac{\mathrm{d}}{\mathrm{d}\mathbf{R}}$,因此$\frac{\mathrm{d}}{\mathrm{d}\mathbf{R}}$算符是反厄米算符,这等价于对于任意两个态,满足
        \begin{equation}
          d_{ij}=\langle\psi_i|\frac{\mathrm{d}}{\mathrm{d}\mathbf{R}}|\psi_j\rangle=-\langle\psi_j|\frac{\mathrm{d}}{\mathrm{d}\mathbf{R}}|\psi_i\rangle=-d_{ji}
          \label{dij=-dji}
        \end{equation}
        根据上式还可以推出一个平凡的结论$d_{ii}=0$。如果回顾等式(\ref{numerical dij})的结果,可以认为在同一个态的前提下分子约等于$1-1$,最终得到$0$。除去数值方法,还有一种常数的解析求解非绝热耦合的思路。如果$|\psi_i\rangle$和$|\psi_j\rangle$是哈密顿算符的两个本征态,显然有
        \begin{equation}
          \begin{aligned}
            0=\frac{\mathrm{d}}{\mathrm{d}\mathbf{R}}\langle\psi_i|\hat{H}|\psi_j\rangle&=E_j d_{ji}+\langle\psi_i|\frac{\mathrm{d}\hat{H}}{\mathrm{d}\mathbf{R}}|\psi_j\rangle+E_i d_{ij}\\
            d_{ij}&=\frac{\langle\psi_i|\frac{\mathrm{d}\hat{H}}{\mathrm{d}\mathbf{R}}|\psi_j\rangle}{E_j-E_i}
          \end{aligned} 
          \label{analytical dij}
        \end{equation}
        如果知道哈密顿量的形式,对核坐标导数的积分可以直接用来计算非绝热耦合。
      \subsection{绝热表象与透热表象}
        本文中所说的绝热(adiabatic)与透热(diabatic)与热学中相关概念并无直接联系,属于量化领域常用的一种表达。绝热是相对于Born–Oppenheimer近似而言的,需要注意由于翻译的问题,请不要混淆非绝热(non-adiabatic)和透热(diabatic)两个概念。

        根据Troy Van Voorhis在他那篇著名的综述\footnote{Annu. Rev. Phys. Chem. 2010. 61:149–170: The Diabatic Picture of Electron Transfer, Reaction Barriers, and Molecular Dynamics}中的说法,绝热态定义为BO近似下的哈密顿量的本征态,也就是说在BO近似下解定态Schr\"odinger方程(电子结构计算)得到的波函数就是绝热态,即$\hat{H}|\phi_i\rangle=E_i|\phi_i\rangle$。而透热态是指不随核坐标(分子构型)变化的态,最常用的就是Troy提到的NaCl的例子,透热态其实是量子化学家们为了理解图像和计算的方便提出的不真实存在的量子态,如不管Na和Cl的距离多大,永远为两个中性原子的态,或者是永远是两个离子的态。这样说来十分抽象,但是本质上就是其波函数不随核坐标变化而变化,一个严格的定义为透热态的非绝热耦合为0(甚至很多时候简化为$\frac{\mathrm{d}}{\mathrm{d}\mathbf{R}}|\phi_j\rangle=0$)。在电荷转移中的图像更加清晰,如果电荷转移发生在两个分子间,透热态通常定为电荷局域在第一个分子上的态和局域在第二个分子上的态,而真实的态(电荷分散在两个分子上,也就是绝热态)是两个透热态的线性组合。

        动力学的计算既可以在绝热表象下进行,也可以在透热表象下进行,但是很多时候为了节省计算量,我们不会在每个时刻都计算对应的能量与NAC,而是用某种函数对其进行拟合,每到一个分子构型,只需要用拟合函数计算即可。尽管在实际操作中,绝热态是很容易得到的,因为绝热态是哈密顿量本征态,我们可以用各种电子结构的计算方法得到他们,但是在绝热表象下,两个绝热态之间存在非绝热耦合,势能面间也会有圆锥交叉或者trivial crossing,在交叉点处的非绝热耦合是一个奇点,通常很难用函数拟合或作其他处理,因此人们很多时候会希望通过绝热态的线性组合来构建透热态,在透热表象下进行动力学模拟,这个过程称为透热化(diabatization)。透热化的方法非常多,本向导中也不作仔细的展开,但是根据Troy的综述,真正的透热态是难以通过绝热态线性组合得到的,所以很多文章中会使用构建半透热态(quasi-diabatic state)的说法。需要注意,根据非绝热耦合等于零的定义,透热态按照某个固定的系数的线性组合还是透热态,也就是透热态并不是唯一的,这种不唯一性给了量子化学家构建透热态的更大的自由度。这也是目前透热化的方法多种多样的一个原因。

        在透热表象下,由于透热态不是哈密顿量的本征态,所以此时哈密顿量不再是对角矩阵,而是有非零的非对角元,在某些问题(如电荷转移)中,这些非对角元通常被叫做转移积分(transfer integrals)或者通常称为透热耦合(diabatic couplings, DCs),对角化透热态哈密顿量应该能重新得到绝热态的能量。在透热表象下,可以认为式(\ref{time-depend-coeff})中的$V_{ij}$就是转移积分,所以在某种程度上说,即使透热化并不完全,仍有很小数值的非绝热耦合,只要该项对转移积分是一个小量,对动力学的影响也可以忽略,但是在绝热表象中由于$V_{ij}=0$,因此非绝热耦合即使有些时候较小,仍是十分重要的。
      \subsection{势间跳跃方法与FSSH}
        势间跳跃(Surface Hopping)方法是一种混合量子经典(Mixed-Quantum-Classical)方法\footnote{需要注意混合量子经典与半经典方法(Semi-Classical)不同,
        前者一般指方法中既有经典部分又有量子部分,后者偏向量子力学做经典近似,演化一个半量子半经典的方程},它是非绝热动力学中非常常用的一类,尽管很早就有类似的
        思路,直到1990年John Tully提出了最少跃迁的势间跳跃(Fewest Switches Surface Hopping, FSSH)方法\footnote{The Journal of Chemical Physics 93, 1061(1990): Molecular dynamics with electronic transitions},在这之后surface hopping方法被大量使用并且一直被发展,王老师课题组的核心研究方向就是SH方法中某些复杂问题的处理以及将其往更大体系更多自由度推广。

        在势间跳跃方法,原子核做经典处理,按照分子动力学的方法运动,但是其感受到的势能是由电子势能而非力场提供,电子的势能是根据电子的Schr\"odinger方程描述的,即电子的运动是量子处理。简单地说,如果读者熟悉分子动力学,那么这套方法理解起来是非常容易的,它的思路非常平凡,即在一个某一个时刻开始,在$dt$的时间间隔内用经典方法演化核的运动,然后计算电子波函数在这个$dt$内的演化,电子势能梯度为原子核的运动提供力,进而演化下一个时刻的原子核运动和电子运动,如此重复,就可以得到非绝热动力学的模拟结果。

        当然尽管图像上非常简单,但是实际操作起来会有各种各样的问题,比如下一小节中稍微涉及的势能面交叉和退相干的问题,包括电子如何在势能面上运动,都是现在仍然在讨论的重要问题,我们都会稍作涉及,但是首先要回到最早的FSSH方法上来。

        之所以称为Surface Hopping,就是因为在这套方法中电子的运动其实是在势能面间的跳跃,在特定的时刻内,电子只会存在于其中的某一个势能面(即通常说的active态),不同的SH方法的核心区别通常在于如何定义电子在不同势能面之间的跃迁概率,什么时候会跃迁。当然如果我们研究一个简单的隧穿问题,单一的SH计算只会得到一个固定的结果,要么穿过了势垒要么被反射等等,不会得到具体的隧穿概率数据,因此SH方法通常需要做很多次轨迹,然后取轨迹平均来得到物理观测量的结果。

        在等式(\ref{time-depend-coeff})中,我们演化波函数的系数,但是为了方便,我们会使用密度矩阵的演化,在等式(\ref{linear combination})的展开形式下,密度矩阵的矩阵元定义为$a_{kj}=c_k c_j^*$,在这个两边取时间导数并将(\ref{time-depend-coeff})的结果代入,我们就得到了关键性的等式
        \begin{equation}
          \begin{aligned} i \hbar \dot{a}_{k j}=& \sum_{T}\left\{a_{l j}\left[V_{k l}-i \hbar \mathbf{R} \cdot \mathbf{d}_{k l}\right]\right.\\ &-a_{k l}\left[V_{l j}-i \hbar \mathbf{R} \cdot \mathbf{d}_{i j}\right] \} \end{aligned}
          \label{time-de DM}
        \end{equation}
        我们通常关心的是密度矩阵的对角元,就是总电子波函数在某个势能面上的\textbf{布居}(\textbf{Population}),令上式中的$k=j$我们得到
        \begin{equation}
          \dot{a}_{k k}=\sum_{l \neq k} b_{k l}
        \end{equation}
        其中
        \begin{equation}
          b_{k l}=2 \hbar^{-1} \operatorname{Im}\left(a_{k l}^{*} V_{k l}\right)-2 \operatorname{Re}\left(a_{k l}^{*} \mathbf{R} \cdot \mathbf{d}_{k l}\right)
          \label{time-de DM diagnol}
        \end{equation}
        而$\dot{a}_{k k}$就是势能面$k$上的布居变化,这里我们引入最少跳跃(Fewest Switches, FS)的思想,即在一个时间间隔$\mathrm{d}t$内的布居变化完全由当前态向其他态的跃迁提供,以两个态的体系为例,如果在时间$t$时,假设所有轨迹中电子当前处在1态的轨迹数为$N_1(t)=a_{11}(t)N$,其中$a_{11}$是$t$时刻电子在1态的布居,$N$为总轨迹数。与之类似,我们有$t$时刻处于2态的轨迹数$N_2(t)=a_{22}(t)N$,在很小的时间间隔$\mathrm{d}t$之后,两个布居为$a_{11}(t+\mathrm{d}t)$和$a_{22}(t+\mathrm{d}t)$,不失一般性,我们假设$a_{11}(t)<a_{11}(t+\mathrm{d}t),a_{22}(t)>a_{22}(t+\mathrm{d}t)$,我们就固定这个时间间隔内从态1跃迁到态2的轨迹数为$a_{11}(t+\mathrm{d}t)-a_{11}(t)$,而从态2跃迁到态1的轨迹数为$0$,原则上可以两者各加一个常数,FS的思想等价于固定这个常数为$0$。这样处理之后,我们得到$\dot{a}_{22}=-\dot{a}_{11}=\frac{a_{11}(t)-a_{11}(t+\mathrm{d}t)}{\mathrm{d}t}$,则如下定义的跃迁概率
        \begin{equation}
          P_{1\rightarrow2}\equiv\frac{a_{11}(t)-a_{11}(t+\mathrm{d}t)}{a_{11}(t)}=\frac{\dot{a}_{22} \mathrm{d}t }{ a_{11}(t)}
        \end{equation}
        如果只有两个能级,将等式(\ref{time-de DM diagnol})代入得到
        \begin{equation}
          P_{1\rightarrow2}=\frac{b_{21} \mathrm{d}t }{ a_{11}(t)}
          \label{hopping probability}
        \end{equation}
        上式就是FSSH的核心公式\footnote{细心的读者可能会发现与Tully原文中的分母不完全相同,但是本文的推导更加合理一点,通常也是使用这一套公式,当然,由于$\mathbf{d}t$内密度矩阵几乎不会变化,究竟是用$t$时刻还是$t+\mathbf{d}t$时刻对结果的影响不大。}。FSSH单个轨迹的流程就是先初始化,赋予研究的体系一个初始动能(动量)和位置,然后每个$\mathbf{d}t$内先用经典力学的方法演化原子核的运动,再利用等式(\ref{hopping probability})计算出当前态跃迁到其他态的概率,用随机数的方法判断是否跃迁,同时还要判断跃迁之后是否满足能量守恒,如果满足,则完成跃迁并修正体系原子核运动的速度,不断重复上述过程就得到了动力学演化的结果。
        
      \subsection{鸟瞰SH方法中的特殊问题}
        本小节将会简要地介绍在SH发展这么多年来,出现的很多问题,包括王老师组学长学姐在这方面的工作的简介,但是由于编者水平有限,只能略作介绍,而王老师课题组的各位学长学姐在这方面有更多可说的内容。

        \paragraph{Trivial Crossing:}在势能面上演化动力学的过程中,势能面通常会有各类的交叉和相互作用,多数可以归为Trivial Crossing或者Avoided Crossing,两者的定义图如下所示
        \begin{center}
          \includegraphics[width = 10cm]{fig/crossing.jpg}
        \end{center}
        图中表示分子1和2之间有相互作用,当它们的势能面交叉时就会产生Avoided Crossing,1和3之间由于距离较远没有相互作用,但是在能量空间下,它们对应的态可能十分接近,这样使得它们的势能面存在交叉,即为Trivial Crossing。另外,Trival Crossing并非只存在于实空间距离很远的两个分子上,即使在同一个分子中,由于轨道之间没有或者相互作用非常小(比如对称性不同的分子轨道,$\sigma$键和$\pi$键之间)但是能量接近,也是可能发生Trivial Crossing的。

        但是由于电子结构计算是按照能量排序的,所以在发生Trivial Crossing的地方会有能级顺序的错误,误将别处毫不相关的能级作为了当前能级,而且根据等式(\ref{analytical dij}),此处的NAC是无法被描述的,通常会由于数值计算时过大而使得跃迁概率接近1,发生不该发生的跃迁,比如从分子1直接跳到分子3。

        为了处理这类问题,王老师发展了很多相关方法,比如SC-FSS和CC-FSSH,后者把常见的Trivial Crossing进行了分类,每一类都有对应的解决方案,基本已经可以处理此类问题,具体内容可以参阅相关文章\footnote{J. Phys. Chem. Lett., 2014, 5 (4), pp 713–719: A Simple Solution to the Trivial Crossing Problem in Surface Hopping}\footnote{J. Phys. Chem. Lett., 2018, 9 (15), pp 4319–4325: Crossing Classified and Corrected Fewest Switches Surface Hopping}\footnote{J. Chem. Phys. 148, 104106 (2018): An efficient solution to the decoherence enhanced trivial crossing problem in surface hopping}。

        \paragraph{超交换问题:}我们继续用上文中的图,同时联想化学里的过渡态理论,如果只有1态和2态之间、2态和3态之间有相互作用,但是1态和3态之间没有,按照SH方法的处理,如果想从1态到3态必须经过2态,但如果此时2态有较高的能量,高于此时原子核运动的动能,如果跃迁发生,就不满足能量守恒,尽管1态和3态之间的势能差可以被原子核动能满足,由于SH的操作方法,系统是不可能从1态到达3态的。但这并非真实情况,在量子力学中存在隧穿行为,因此实际上体系是会从1态到达3态的,SH这个错误的来源是忽略了原子核的量子效应导致的。当然,这不意味着SH不能处理超交换问题,王老师提出了在Liouville空间下Surface Hopping\footnote{J. Phys. Chem. Lett., 2015, 6 (19), pp 3827–3833: Fewest Switches Surface Hopping in Liouville Space},可以很好地处理超交换和隧穿的问题。

        \paragraph{退相干:}\footnote{J. Chem. Phys. 134, 244114 (2011): Decoherence and surface hopping: When can averaging over initial conditions help capture the effects of wave packet separation?}The most prominent shortcoming of surface-hopping is that FSSH suffers from the problem of “over-coherence.” To understand how this problem arises, recall that surfacehopping requires equations of motion for both classical nuclear motion and (quantum-mechanical) electronic amplitude propagation. There is an important question here as to whether these equations will always be consistent, in the sense that the electronic state and the nuclear forces correspond correctly to each other. The fully quantum-mechanical Schrodinger equation is, of course, automatically consistent between nuclei and electrons, but semiclassical surfacehopping techniques need not be so. In general, for a swarm of trajectories, FSSH finds consistency by forcing stochastic hops between surfaces so as to match the number of trajectories on each surface with the electronic wavefunction. The result is a set of discontinous classical trajectories and a very reasonable set of populations.

        Now, the FSSH algorithm faces a unique obstacle when a classical nucleus enters a region of configuration space where different surfaces feel very different forces. Ideally, the classical nuclei in FSSH should represent heavy, compact quantum nuclear wavefunctions and these wavefunctions can certainly bifurcate into more than one wave packet on different surfaces, as one would find with exact quantum dynamics or multiple-spawning trajectories.6, 7 While this is clearly the correct physical picture, unfortunately, from the perspective of the electronic degrees of freedom, FSSH does not correctly treat this bifurcation of a nuclear wave packet: instead, the electronic wavefunction is still propagated along the instantaneous nuclear trajectory, as if the electronic amplitudes were unaware that the total wave packet is now breaking apart. In the language of the decoherence literature, the components of the electronic wavefunction on different electronic surfaces always remain coherent. This overcoherence can lead to spurious results for long-time dynamics, especially if nuclei visit more than one distinct region of derivative coupling.
  \end{document} 

